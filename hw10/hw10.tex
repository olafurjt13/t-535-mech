\documentclass[11pt,a4paper,titlepage]{article}
\usepackage[utf8]{inputenc}
\usepackage[english]{babel}
\usepackage[T1]{fontenc}

\RequirePackage[layout=inline]{fixme}

\usepackage{float}
\usepackage{graphicx}
\usepackage{setspace}
\usepackage{amsmath}
\usepackage{courier}
\usepackage{amsmath}
\usepackage{listings}
\usepackage{color}
\usepackage[toc, page]{appendix}

\usepackage{algpseudocode}
\usepackage[bottom]{footmisc}
\usepackage{verbatimbox}

\usepackage{changepage}

\usepackage{multirow}


\definecolor{mygreen}{rgb}{0,0.6,0}
\definecolor{mygray}{rgb}{0.5,0.5,0.5}
\definecolor{mymauve}{rgb}{0.58,0,0.82}


%% Units:
\newcommand{\W}{\,\textrm{W}}
\newcommand{\A}{\,\textrm{A}}
\newcommand{\mA}{\,\textrm{mA}}
\newcommand{\N}{\,\textrm{N}}
\newcommand{\Hz}{\,\textrm{Hz}}
\newcommand{\V}{\,\textrm{V}}
\newcommand{\Ohms}{\,\Omega}
\newcommand{\kOhm}{\,\text{k}\Omega}
\newcommand{\nF}{\,\textrm{nF}}
\newcommand{\dB}{\,\textrm{dB}}
\newcommand{\VperBit}{\,\textrm{V/bit}}
\newcommand{\NperBit}{\,\textrm{N/bit}}

\newcommand{\degC}{\,^{\circ}\text{C}}

\lstset{ %
	backgroundcolor=\color{white},   % choose the background color
	basicstyle=\scriptsize,        % size of fonts used for the code
	breaklines=true,                 % automatic line breaking only at whitespace
	captionpos=b,                    % sets the caption-position to bottom
	commentstyle=\color{mygreen},    % comment style
	escapeinside={\%*}{*)},          % if you want to add LaTeX within your code
	keywordstyle=\color{blue},       % keyword style
	stringstyle=\color{mymauve},     % string literal style
	numbers=left,
}

%\renewcommand{\thesubsection}{\thesection.\alph{subsection}}



\usepackage{booktabs}
\usepackage[backend=biber, bibencoding=utf8, style=ieee]{biblatex}

\addbibresource{references.bib}
\usepackage[final,hidelinks]{hyperref} % must be last package loaded

\author{Ólafur Jón Thoroddsen}  % My name, for the titlepage
\title{\includegraphics{graphics/ru-logo}\\\vspace{10mm}
	Mechatronics II\\T-535-MECH \ \\Homework 10}  % The title, for the titlepage

\begin{document}
	\pagenumbering{arabic}
	\maketitle
	
	\tableofcontents
	\pagebreak
	
	\section{ATmega328p memory types}
	
	There are three types of memory on board the ATmega328p microcontroller, Flash, RAM and EEPROM. They all differ in size and function.
	
	\begin{table}[H]
		\centering
		\begin{tabular}{ccc}
			\toprule
			Flash	&	EEPROM	&	RAM	\\
			\midrule
			32 kB	&	1 kB	&	2kB	\\
			\bottomrule
		\end{tabular}
		\caption{The memory types of the ATmega328p and the sizes of each one.}
		\label{tab:memory}
	\end{table}
	
	\subsection{Flash memory}
	The Flash memory is the largest memory on board. It is where programs are loaded into when programming  the microcontroller.
	
	\subsection{EEPROM}
	\textsc{EEPROM} is an abbreviation for Electrically Erasable Programmable Read Only Memory. It is non-volatile memory which means that it is not affected when the processor itself resets or is powered down. It has a limited number of write cycles which means that it should be taken into account in programming to limit the number of \textsc{EEPROM} writes as possible.
	
	\subsection{RAM}
	The \textsc{RAM} is an abbreviation for Random Access Memory and it is where a program stores local variables at runtime in a stack, growing from bottom up. The \textsc{RAM} also is the place where the Stack is stored, which is where subroutines store their local variables.
	
	
	\section{Storing program parameters between sessions}
	The way to store program parameters between sessions is to use the \textsc{EEPROM}, which will not get erased when the microcontroller is shut down or is reset. Any parameters can be written and reloaded from the \textsc{EEPROM}.
	
	
	\section{Power supply design for an Arduino}
	The Arduino needs  7-12\V DC voltage to operate. For a microcontroller or any such delicate device, it is desirable to have the input voltage as stable as possible. To be able to power the Arduino from the main household power, which is 230\V AC, a power adapting circuit is needed. The following describes a 230\V AC to 9\V DC transformer (9\V because it is in the middle of the desirable input voltage range).
	
	\begin{enumerate}
		\item[Stage 1:]
		To step the main voltage down, a 230\V AC to 12\V AC transformer is used, it is good to not step the voltage all the way down yet because of possible power loss in the next stages.
		\item[Stage 2:]
		Full Bridge Rectification of the AC signal. A diode bridge can be used to rectify the AC signal, single diode rectification is also possible but that means that half of the signal gets lost and the power is not used as efficiently.
		\item[Stage 3:]
		After rectification, the signal is still very rapidly changing and a smoothing capacitor is needed to minimize the fluctuations.
		\item[Stage 4:]
		Because the world is not perfect and capacitors leak voltage, a voltage regulator is needed to  reduce  ripple voltage and step the DC voltage down to the target 9\V DC. Possible choices for a linear voltage regulator (linear tend to be more accurate, which is what we want) are for example the Farchild LM7809CT 9\V\ regulator with a 2\V\ typical dropout voltage at 1\A\, the Texas Instruments LM2940T-9.0/NOPB with a typical dropout voltage of 0.5\V\ at 1\A\ and only 0.11\V\ at 100\mA . A third choice could be the ST Microelectronics L7809CV which has a dropout voltage of 2\V\ at 1\A\ like the LM7809CT.
		
		For this transformer, it is not necessarily required to have a very low dropout voltage. With these three choices, the L7809CV is the most inexpensive so it should be the final choice.
		\item[Step 5:] A final Low-Pass Filtering is applied to before the output to ensure a stable voltage.
	\end{enumerate}
	
	\begin{figure}[H]
		\centering
		\includegraphics[width=\textwidth]{graphics/transformer2}
		\caption{A 230VAC to 9VDC power adapting circuit}
		\label{fig:transformer}
	\end{figure}
	
	\pagebreak
	\section{Progress with my project}
	
	\subsection{Last few weeks}
	
	These last few weeks have been pivotal in the progress of the project. The SPI communication with the IMU sensor is finally working and from there it was possible to create useful tools for signal processing.
	Read functions were implemented for acceleration (\verb|readAcc()|) and angular rate (\verb|readGyro()|) data from the IMU that use a moving average filter with variable smoothness to reduce random noise in the signals.
	The filter is designed essentially as a queue with a finite number of spots $N$, when a new measurement is read, it causes the first measurement to leave the queue to make room, then the average of the $N$ values is taken and returned. This is illustrated in figure~\ref{fig:movingavg}.
	 
	 \begin{figure}[H]
		\centering
		\includegraphics[width=\textwidth]{graphics/movingAvgFilter}
		\caption{The idea behind the moving average filter. As each new measurement is taken, the oldest measurement is discarded and the new one gets a place in a buffer, the average is then calculated and returned}
		\label{fig:movingavg}
	 \end{figure}
	 
	  \noindent A calibration function was implemented which integrates the error of each value (X,Y,Z acceleration and angular rate) and then that is subtracted from subsequent measurements from the sensor.
	  
	  \noindent Work was initiated on the motor control system which is being simultaneously designed and implemented using the IMU sensor data as its inputs. This will be a focus of next weeks work. 

	\subsection{Next week}
	
	\begin{table}[H]
		\centering
		\begin{tabular}{llc}
			\toprule
			Task no.	&	Task	&	ETC\footnotemark\\
			\midrule
			1	&	\begin{tabular}{@{}l@{}}Write the PID control loop\\ for the motor control system\\\end{tabular} &	10 hours \\	
			\midrule	
			2	&	\begin{tabular}{@{}l@{}}Integrate the IMU sensor data\\ into the control system\end{tabular}	& 10 hours	\\
			\midrule
			3	&	\begin{tabular}{@{}l@{}}Create a PCB for placing all\\ components inside the robot\end{tabular}	& 10 hours\\
			\midrule
			4	&	\begin{tabular}{@{}l@{}}Design the final look of\\ the robot in CAD/Illustrator\end{tabular}	& 5 hours\\
			\bottomrule
		\end{tabular}
		\label{tab:nextweek}
	\end{table}
	\footnotetext{Estimated Time to Complete}
	
	\subsection{Long term plan}
	
	\begin{table}[H]
		\centering
		\hspace*{-2cm}
		\begin{tabular}{lccc}
			\toprule
			Week	&	Software design	&	Mechanical design	&	Testing\\
			\midrule
			11	&	\begin{tabular}{@{}l@{}}Integrate IMU, PID\\ and PWM modules\end{tabular}	&	Altium schematics of electronics	&	Integration\\
			\midrule
			12	&	Integration	&	Integration	&	Integration	\\
			\bottomrule
		\end{tabular}
		\label{tab:longterm}
		\hspace*{-2cm}
	\end{table}
	
	
	
\pagebreak
%\section*{Appendices}
\appendix



\pagebreak
\printbibliography

\end{document}