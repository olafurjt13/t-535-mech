\documentclass[11pt,a4paper,titlepage]{article}
\usepackage[utf8]{inputenc}
\usepackage[english]{babel}
\usepackage[T1]{fontenc}

\RequirePackage[layout=inline]{fixme}

\usepackage{float}
\usepackage{graphicx}
\usepackage{setspace}
\usepackage{amsmath}
\usepackage{courier}
\usepackage{amsmath}
\usepackage{listings}
\usepackage{color}
\usepackage[toc, page]{appendix}

\usepackage{algpseudocode}
\usepackage[bottom]{footmisc}
\usepackage{verbatimbox}

\usepackage{changepage}

\usepackage{multirow}


\definecolor{mygreen}{rgb}{0,0.6,0}
\definecolor{mygray}{rgb}{0.5,0.5,0.5}
\definecolor{mymauve}{rgb}{0.58,0,0.82}


%% Units:
\newcommand{\W}{\,\textrm{W}}
\newcommand{\A}{\,\textrm{A}}
\newcommand{\mA}{\,\textrm{mA}}
\newcommand{\N}{\,\textrm{N}}
\newcommand{\Hz}{\,\textrm{Hz}}
\newcommand{\V}{\,\textrm{V}}
\newcommand{\Ohms}{\,\Omega}
\newcommand{\kOhm}{\,\text{k}\Omega}
\newcommand{\nF}{\,\textrm{nF}}
\newcommand{\dB}{\,\textrm{dB}}
\newcommand{\VperBit}{\,\textrm{V/bit}}
\newcommand{\NperBit}{\,\textrm{N/bit}}

\newcommand{\degC}{\,^{\circ}\text{C}}

\lstset{ %
	backgroundcolor=\color{white},   % choose the background color
	basicstyle=\scriptsize,        % size of fonts used for the code
	breaklines=true,                 % automatic line breaking only at whitespace
	captionpos=b,                    % sets the caption-position to bottom
	commentstyle=\color{mygreen},    % comment style
	escapeinside={\%*}{*)},          % if you want to add LaTeX within your code
	keywordstyle=\color{blue},       % keyword style
	stringstyle=\color{mymauve},     % string literal style
	numbers=left,
}

%\renewcommand{\thesubsection}{\thesection.\alph{subsection}}



\usepackage{booktabs}
\usepackage[backend=biber, bibencoding=utf8, style=ieee]{biblatex}

\addbibresource{references.bib}
\usepackage[final,hidelinks]{hyperref} % must be last package loaded

\author{Ólafur Jón Thoroddsen}  % My name, for the titlepage
\title{\includegraphics{graphics/ru-logo}\\\vspace{10mm}
	Mechatronics II\\T-535-MECH \ \\Homework 10}  % The title, for the titlepage

\begin{document}
	\pagenumbering{arabic}
	\maketitle
	
	\tableofcontents
	\pagebreak
	
	\section{ATmega328p memory types}
	
	There are three types of memory on board the ATmega328p microcontroller, Flash, RAM and EEPROM. They all differ in size and function.
	
	\begin{table}[H]
		\centering
		\begin{tabular}{ccc}
			\toprule
			Flash	&	EEPROM	&	RAM	\\
			\midrule
			32 kB	&	1 kB	&	2kB	\\
			\bottomrule
		\end{tabular}
		\caption{The memory types of the ATmega328p and the sizes of each one.}
		\label{tab:memory}
	\end{table}
	
	\subsection{Flash memory}
	The Flash memory is the largest memory on board. It is where programs are loaded into when programming  the microcontroller.
	
	\subsection{EEPROM}
	\textsc{EEPROM} is an abbreviation for Electrically Erasable Programmable Read Only Memory. It is non-volatile memory which means that it is not affected when the processor itself resets or is powered down. It has a limited number of write cycles which means that it should be taken into account in programming to limit the number of \textsc{EEPROM} writes as possible.
	
	\subsection{RAM}
	The \textsc{RAM} is an abbreviation for Random Access Memory and it is where a program stores local variables at runtime in a stack, growing from bottom up. The \textsc{RAM} also is the place where the Stack is stored, which is where subroutines store their local variables.
	
	
	\section{Storing program parameters between sessions}
	The way to store program parameters between sessions is to use the \textsc{EEPROM}, which will not get erased when the microcontroller is shut down or is reset.
	
	
	\section{Power supply design for an Arduino}
	A power supply for an Arduino needs to take the main voltage of the household, 220VAC in Europe
	
	
	\pagebreak
	\section{Progress with my project}
	
	\subsection{Last week}
	
	
	

	\subsection{Next week}
	
	\begin{table}[h]
		\centering
		\begin{tabular}{llc}
			\toprule
			Task no.	&	Task	&	ETC\footnotemark\\
			\midrule
			1	&	\begin{tabular}{@{}l@{}}Finish the IMU module\\\end{tabular} &	5 hours \\	
			\midrule	
			2	&	\begin{tabular}{@{}l@{}}Write the PWM code that uses output compare pins\end{tabular}	&	10 hours\\
			\midrule
			3	&	\begin{tabular}{@{}l@{}}Test the accuracy of the\\IMU sensor\end{tabular}	&	5 hours\\
			\midrule
			4	&		\begin{tabular}{@{}l@{}}Plan the control system and\\start designing at a high level\end{tabular}	&	5 hours\\
			\bottomrule
		\end{tabular}
		\label{tab:nextweek}
	\end{table}
	
	
	\footnotetext{Estimated Time to Complete}

	\subsection{Long term plan}
	
	\begin{table}[h]
		\centering
		\hspace*{-2cm}
		\begin{tabular}{lccc}
			\toprule
			Week	&	Software design	&	Mechanical design	&	Testing\\
			\midrule
			9	&	IMU \& PWM	&	\begin{tabular}{@{}l@{}}Power circuitry\\2nd prototype\end{tabular}	&	\begin{tabular}{@{}l@{}}Estimate power consumption\\PID motor control\end{tabular}\\
			\midrule
			10	&	Rethink PID control	&	3D drawing of the robot	&	Power consumption	\\
			\midrule
			11	&	\begin{tabular}{@{}l@{}}Integrate IMU, PID\\ and PWM modules\end{tabular}	&	Altium schematics of electronics	&	Integration\\
			\midrule
			12	&	Integration	&	Integration	&	Integration	\\
			\bottomrule
		\end{tabular}
		\label{tab:longterm}
		\hspace*{-2cm}
	\end{table}
	
	
	
\pagebreak
%\section*{Appendices}
\appendix



\pagebreak
\printbibliography

\end{document}